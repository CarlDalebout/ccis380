%-*-latex-*-
\textsc{Field axioms and axioms of $\R$}

$(F, 0, 1, +, \cdot)$ is a \defonet{field} if
$F$ is a set and $0,1 \in F$ and satisfies the following:
\begin{enumerate}
  \li \textsc{Axioms of $+$}
  \begin{myenumt}
    \li \textsc{Closure of $+$}: If $x, y \in \R$, then $x + y \in \R$.
    \li \textsc{Associativity of $+$}: If $x, y, z \in \R$, then
    $(x + y) + z = x + (y + z)$.
    \li \textsc{Inverse of $+$}: If $x \in \R$, then
    there is some $x' \in \R$ such that $x + x' = 0 = x' + x$.
    (It can be shown that additive inverse of $x$ is unique.
    Therefore one can denote the additive inverse of $x$ by $-x$.)
    \li \textsc{Neutrality of $+$}:
    If $x \in \R$, then $x + 0 = x = 0 + x$.
    \li \textsc{Commutativity of $+$}:
    If $x, y \in \R$, then $x + y = y + x$.
  \end{myenumt}
  We define $x - y = x + (-y)$ if $x, y \in F$.
  \li \textsc{Axioms of $\cdot$}
  \begin{myenumt}
    \li \textsc{Closure of $\cdot$}: If $x, y \in \R$, then $x \cdot y \in \R$.
    \li \textsc{Associativity of $\cdot$}: If $x, y, z \in \R$, then
    $(x \cdot y) \cdot z = x \cdot (y \cdot z)$.
    \li \textsc{Inverse of $\cdot$}: If $x \in \R$ and $x \neq 0$, then
    there is some $x' \in \R$ such that $x \cdot x' = 1 = x' \cdot x$.
    (It can be shown that multiplicative inverse of $x$ is unique.
    Therefore one can denote the additive inverse of $x$ by $x^{-1}$.)
    \li \textsc{Neutrality of $\cdot$}:
    If $x \in \R$, then $x \cdot 1 = x = 1 \cdot x$.
    \li \textsc{Commutativity of $\cdot$}:
    If $x, y \in \R$, then $x \cdot y = y \cdot x$.
  \end{myenumt}
  Frequently we'll write $xy$ for $x \cdot y$.
  Also, if $y \neq 0$, I'll write $x/y$ for $x \cdot y^{-1}$.
  \li \textsc{Distributivity}:
  If $x, y, z \in \R$, then $x \cdot (y  + z) = x \cdot y + x \cdot z \in \R$.
\end{enumerate}
We will assume the following axioms about $(\R, 0, 1, +, \cdot)$:
\begin{enumerate}
  \li $(\R, 0, 1, + \cdot)$ is a field, i.e. it satisfies the field axioms.
  \li \textsc{Nontriviality}: $0 \neq 1$.
  \li \textsc{Order axioms}: There is a subset $P \subseteq \R$ such that
  \begin{myenumt}
    \li If $x \in \R$, exactly one of the following is true:
    $x \in P$, $x = 0$, $-x \in P$
    \li If $x, y \in P$, then $x + y \in P$.
    \li If $x, y \in P$, then $x \cdot y \in P$.
  \end{myenumt}
  With the above, I can define $x > 0$ is $x \in P$
  and also $x \geq 0$ if $x = 0$ or $x > 0$.
  Furthermore I can define $x > y$ is $x - y > 0$ and $x \geq y$ if
  $x - y \geq 0$.
  \li \textsc{Least upper bound}:
  Every nonempty set of $\R$ has a least upper bound.
  (If $X$ is a nonempty set of $\R$, then $m \in \R$ is an \defonet{upper bound}
  of
  $X$ if
  \[
  x \leq m \text{ for all $x \in X$}
  \]
  (duh).
  $\ell \in \R$ is a \defonet{least upper bound (l.u.b.)} of
  $X$ if $\ell$ is an upper bound of $X$
  and if $m$ is an upper bound of $X$, then $\ell \leq m$.)
\end{enumerate}
For $x \in \R$, the absolute value of $x$ is defined to be
\[
|x|
=
\begin{cases}
  x  & \text{ if $x \geq 0$} \\
  -x & \text{ otherwise}
\end{cases}
\]

The following facts about any field $F$ can be proven from the field axioms of
$F$.
But we won't prove them.
We will simply assume that they hold.
You can try to prove them on your own. Most of these are proven in CISS451.

\textbf{Proposition A.}
Let $a, b, c, d \in F$.
\begin{myenumt}
\item $-0 = 0$
\item $0 \cdot a = 0 = a \cdot 0$
\item $(-1) \cdot a = -a = a \cdot (-1)$
\item $(-1) \cdot (-1) = 1$
\end{myenumt}

Here are some basic facts about the absolute value function on $\R$.

\textbf{Proposition B.}
Let $a, b, c, d \in \R$.
\begin{myenumt}
\item $|0| = 0$
\item \textsc{Multiplicativity}: $|xy| = |x| |y|$
\item \textsc{Triangle inequality}: $|x + y| \leq |x| + |y|$
\end{myenumt}

Most of the above is not difficult.
The only one that might not be immediate is the triangle inequality of
$\R$.
It's a good exercise.

In the proofs below, you can quote Proposition A and Proposition B.
You have to specify which part.
For instance
\begin{align*}
x + ((-1) \cdot (-1)) \cdot y
&= x + 1 \cdot y  & & \text{ by Proposition A(d)} \\
&= x + y  & & \text{ by neutral axiom of $\cdot$} \\
\end{align*}
